% 
% Annual CCN conference
% Sample LaTeX Two-Page Summary -- Proceedings Format
% based on the prior cognitive science style file
\documentclass[10pt,letterpaper]{article}

\usepackage{ccn}
\usepackage{pslatex}
\usepackage{apacite}


\title{Efficient vs. Inefficient Sub-optimality in Human Sequential Planning}
 
\author{{\large \bf Zahy Bnaya (zahy.bnaya@nyu.edu)} \\
Center For Neural Science, New York University\\
  \AND {\large \bf Weiji Ma (weiji.ma@nyu.edu)} \\
Center For Neural Science, New York University\\
}

\begin{document}

\maketitle


\begin{quote}
\small
\textbf{Keywords:} 
decision making; human behavior; planning; heuristic search
\end{quote}

\section{Planning in Fully Observable Deterministic Environments}

Sequential planning is a difficult task even when the environment is deterministic and fully observable.
State space search is a successful method to solve such tasks.  
On simple cases, the optimal solution can be found using simple uninformed search such as Breadth-First-Search or Depth-First-Search. 
Tasks with larger state space requires using heuristic search algorithms such as A* (Hart and Nillsen) or IDA* (Korf). When the heuristic function is admissible, the algorithm is guaranteed to find the optimal solution.


\subsection{Efficient and Inefficient sub-optimality}

Humans are not optimal planners.  Therefore we examine "satisficing planning" methods. These methods efficiently sacrifice optimality in favor of tractability of larger tasks. While search methods are principled, it is not clear how much of the sub-optimality presented by humans stems from efficient manipulation of their limited computational abilities.

There are several ways to effectively achieve sub-optimal solutions to complex tasks. 

\begin{itemize}
	\item Partial search, such as online-search (Rich Korf) or agent-centered search  (Sven Konning).
	\item Hierarchical search (Mainly Dana Nau. and the other guy). 
	\item Bounded suboptimal algorithms such as WA* algorithms or using inadmissible heuristic functions.
\end{itemize}

Usually, these methods show significant increase in solution capability without dramatically hurting the solution quality; some of these methods also have strict quality bounds. 
Alternatively, suboptimal solutions can also be caused by mechanisms that are not necessarily efficient in trading-off optimality and computational efforts such as-

\begin{itemize}
	\item Lapsing to random moves  (can be actually efficient, I guess. Check RT?).
	\item Adding Noise to the heuristic value (can actually sometimes work (Nathan's work?) on domains with a lot of local minima).
	\item Unprincipled prioritizing of the search queue.
\end{itemize}

While it is easy to find motivation for the first category of factors, the latter category of factors is not necessarily justified. 

\subsection{Efficient Heuristic Extraction}
In a similar way there are generally two categories of effectively extracting heuristic functions from a planning domain.  
\begin{enumerate}
\item {\bf Relaxation} - ignoring a subset of the constraints of the domain . 
\item {\bf Abstraction} - grouping available actions or states together. 
\end{enumerate}


We use the game of rush-hour (\url{http://rush-hour.com}) as a testbed.
We compare two families of models - "efficient" suboptimal planners and "inefficient" suboptimal planners.  
Both families use one of the following search techniques - (a) regular forward search (b) hierarchical search  (c) partial search (RTA*/LRTA*)
We also suggest the following heuristic functions:  (a) uninformed (b) abstraction - "MagSize" (c) relaxation - (passing through cars) (d) minimin pruning. 

For the "inefficient" family we also add - lapse, search-lapse, and noise on heuristic function.  

To measure the effectiveness of search solvers we observe the solution quality, and the number of expanded and generated nodes, (and p-time-per-node?)

We ask the following questions: How much of suboptimality can be explained by efficient suboptimality vs. Inefficient suboptimality?
What heuristic function is more efficient in our case?
What is more similar to the way our subjects had played?

Show correlation of RT and number of nodes of models.
Solution Quality of models and subjects. 


\section{Result}
We asked 10 subjects to solve up to 40 instances of the game rush-hour.



\fbox{\begin{minipage}{17em}
{\bf Figure 1: Human solution quality (histograms and stacked bars per instance)}\end{minipage}}\\

Our subjects solved the puzzle instances with up to 30\% more steps on average than the possible shortest solution. (add Fig optimality)


\fbox{\begin{minipage}{17em}
{\bf Figure 2: Human response times (per instance)}\end{minipage}}\\


\fbox{\begin{minipage}{17em}
{\bf Figure 3: Comparing the "effectiveness" of our models. (Solution Quality and Nodes expanded/generated)  (different h-functions, RTA* vs LRTA*  }\\
\end{minipage}}\\

\fbox{\begin{minipage}{17em}
{\bf Figure 4: same as Fig3, now with "inefficient" suboptimality}\\
\end{minipage}}\\

Correlation between RT, and search quality to models with effective suboptimality.
Correlation between RT and models with ineffective suboptimality.

We show correlation between the solution time of instances and the number of expanded nodes when the heuristic function is XX. We show higher correlation of what subjects were doing and the hierarchical search. We show that heuristic functions based on abstraction are more likely than heuristic functions based on relaxation. 


\section{Conclusions}

\begin{itemize}
	\item X\% of human sub-optimality can be explained by effective sub-optimality. 
	\item Humans use hierarchical planning (because of correlation between RT and subgoals).
	\item Humans use abstraction heuristic (because of corrections between RT and expanded nodes on these heuristics).
\end{itemize}



%#Summary reports:
%#1.Stacked bars of solved/skipped/restarted
%#2.Stacked bars of response times/ human lengths.
%#3.Solving time per instance histogram
%#4.Solving time as a relation of expanded nodes
%#5.RT per move for all moves and all subjects
%#6.Distance from the goal vs move number (plot per instance, SEM of subjects)
%#7.Distance from the goal vs move number (plot per subject, SEM of instances)
%#8.RT vs move number (plot per instance)
%#9.num of backtracks per instances
%#10. RT for subgoals and non-subgoals (find specific states with high RT)
%#11. RT vs distance to goal
%#12. RT vs number of legal moves
%#13. RT vs change in distance (delta) three histograms
%#
%"""






%%The entire contribution of a short summary submission (including
%figures, references, and anything else) can be no longer than two
%pages. This short summary format is to be used for workshop and
%tutorial descriptions, symposia summaries, and publication-based
%presentation extended abstracts. Unlike submitted research papers,
%short summary submissions should \emph{not} begin with a separate
%abstract. Prior to the first section of the short summary, there
%should be the header ``{\bf Keywords:}'' followed by a list of
%descriptive keywords separated by semicolons, all in 9~point font, as
%shown above.
%
%The text of the paper should be formatted in two columns with an
%overall width of 7 inches (17.8 cm) and length of 9.25 inches (23.5
%cm), with 0.25 inches between the columns. Leave two line spaces
%between the last author listed and the text of the paper. The left
%margin should be 0.75 inches and the top margin should be 1 inch.
%\textbf{The right and bottom margins will depend on whether you use
%  U.S. letter or A4 paper, so you must be sure to measure the width of
%  the printed text.} Use 10~point Modern with 12~point vertical
%spacing, unless otherwise specified.
%
%The title should be in 14~point, bold, and centered. The title should
%be formatted with initial caps (the first letter of content words
%capitalized and the rest lower case). Each author's name should appear
%on a separate line, 11~point bold, and centered, with the author's
%email address in parentheses. Under each author's name list the
%author's affiliation and postal address in ordinary 10~point type.
%
%Indent the first line of each paragraph by 1/8~inch (except for the
%first paragraph of a new section). Do not add extra vertical space
%between paragraphs.
%
%
%\section{First Level Headings}
%
%First level headings should be in 12~point, initial caps, bold and
%centered. Leave one line space above the heading and 1/4~line space
%below the heading.
%
%
%\subsection{Second Level Headings}
%
%Second level headings should be 11~point, initial caps, bold, and
%flush left. Leave one line space above the heading and 1/4~line
%space below the heading.
%
%
%\subsubsection{Third Level Headings}
%
%Third level headings should be 10~point, initial caps, bold, and flush
%left. Leave one line space above the heading, but no space after the
%heading.
%
%
%\section{Formalities, Footnotes, and Floats}
%
%Use standard APA citation format. Citations within the text should
%include the author's last name and year. If the authors' names are
%included in the sentence, place only the year in parentheses, as in
%\citeA{NewellSimon1972a}, but otherwise place the entire reference in
%parentheses with the authors and year separated by a comma
%\cite{NewellSimon1972a}. List multiple references alphabetically and
%separate them by semicolons
%\cite{ChalnickBillman1988a,NewellSimon1972a}. Use the
%``et~al.'' construction only after listing all the authors to a
%publication in an earlier reference and for citations with four or
%more authors.
%
%
%\subsection{Footnotes}
%
%Indicate footnotes with a number\footnote{Sample of the first
%footnote.} in the text. Place the footnotes in 9~point type at the
%bottom of the column on which they appear. Precede the footnote block
%with a horizontal rule.\footnote{Sample of the second footnote.}
%
%
%\subsection{Tables}
%
%Number tables consecutively. Place the table number and title (in
%10~point) above the table with one line space above the caption and
%one line space below it, as in Table~\ref{sample-table}. You may float
%tables to the top or bottom of a column, or set wide tables across
%both columns.
%
%\begin{table}[!ht]
%\begin{center} 
%\caption{Sample table title.} 
%\label{sample-table} 
%\vskip 0.12in
%\begin{tabular}{ll} 
%\hline
%Error type    &  Example \\
%\hline
%Take smaller        &   63 - 44 = 21 \\
%Always borrow~~~~   &   96 - 42 = 34 \\
%0 - N = N           &   70 - 47 = 37 \\
%0 - N = 0           &   70 - 47 = 30 \\
%\hline
%\end{tabular} 
%\end{center} 
%\end{table}
%
%
%\subsection{Figures}
%
%Make sure that the artwork can be printed well (e.g. dark colors) and that 
%the figures make understanding the paper easy.
% Number figures sequentially, placing the figure
%number and caption, in 10~point, after the figure with one line space
%above the caption and one line space below it, as in
%Figure~\ref{sample-figure}. If necessary, leave extra white space at
%the bottom of the page to avoid splitting the figure and figure
%caption. You may float figures to the top or bottom of a column, or
%set wide figures across both columns.
%
%\begin{figure}[ht]
%\begin{center}
%\fbox{CCN figure}
%\end{center}
%\caption{This is a figure.} 
%\label{sample-figure}
%\end{figure}
%
%
%\section{Acknowledgments}
%
%Place acknowledgments (including funding information) in a section at
%the end of the paper.
%
%
%\section{References Instructions}
%
%Follow the APA Publication Manual for citation format, both within the
%text and in the reference list, with the following exceptions: (a) do
%not cite the page numbers of any book, including chapters in edited
%volumes; (b) use the same format for unpublished references as for
%published ones. Alphabetize references by the surnames of the authors,
%with single author entries preceding multiple author entries. Order
%references by the same authors by the year of publication, with the
%earliest first.
%
%Use a first level section heading, ``{\bf References}'', as shown
%below. Use a hanging indent style, with the first line of the
%reference flush against the left margin and subsequent lines indented
%by 1/8~inch. Below are example references for a conference paper, book
%chapter, journal article, dissertation, book, technical report, and
%edited volume, respectively.
%
%\nocite{ChalnickBillman1988a}
%\nocite{Feigenbaum1963a}
%\nocite{Hill1983a}
%\nocite{OhlssonLangley1985a}
%% \nocite{Lewis1978a}
%\nocite{Matlock2001}
%\nocite{NewellSimon1972a}
%\nocite{ShragerLangley1990a}
%

\bibliographystyle{apacite}

\setlength{\bibleftmargin}{.125in}
\setlength{\bibindent}{-\bibleftmargin}

\bibliography{sequential_planning}


\end{document}
